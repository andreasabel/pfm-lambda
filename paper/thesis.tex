\documentclass{article}

\usepackage{fontspec}
\usepackage[backend=biber]{biblatex}
\usepackage[hidelinks]{hyperref}
\usepackage{mathtools,amssymb,amsthm}
\usepackage[llbracket,rrbracket]{stmaryrd}
\usepackage{tikz}
\usepackage{mathpartir}
\usepackage{tabularx}
\usepackage{booktabs}
\usepackage{listings}

\bibliography{sources.bib}

\newtheorem{theorem}{Theorem}
\newtheorem{lemma}[theorem]{Lemma}

\newcommand{\lock}{
  \text{\tikz[baseline]{
      \fill[rounded corners=.1ex] (-.75ex,0) rectangle (.75ex,1ex);
      \draw[line width=.3ex] (-.4ex,.5ex) -- ++(0,.75ex) arc (180:0:.4ex) -- ++(0,-.75ex);
}}}

\DeclareMathOperator\Rpl{Rpl}
\DeclareMathOperator\unbox{unbox}

\begin{document}

\begin{titlepage}\centering

{\scshape\LARGE Master Thesis Half-Time Report\\}

\vspace{0.5cm}

{\huge\bfseries A Parametric Fitch-Style Modal Lambda Calculus\\}

\vspace{2cm}

{\Large Axel Forsman \texttt{<foraxel@student.chalmers.se>} \\}

\vspace{1.0cm}

{\large Supervisor: Nachiappan Valliappan \\}

\vspace{1.0cm}

{\large Examiner: Andreas Abel \\}

\vspace{1.5cm}

{\large Relevant completed courses:\par}

{\itshape \begin{itemize}
  \item DAT060, Logic in Computer Science; and
  \item DAT350, Types for Programs and Proofs.
  \end{itemize}}

\vfill
{\large April 17, 2023 \\}
\end{titlepage}

\section{Introduction}

The \emph{necessity modality}, denoted by $\Box$, where the focus will lie,
has been applied to model confidentiality in information-flow control,
compartmental purity in functional languages,
and more.
For a formula $\phi$, the formula $\Box \phi$ reads
``It is necessarily true that $\phi$'' -
$\Box$ changes the \emph{mode} of $\phi$.
So called Fitch-style modal deduction,
where modalities are eliminated by opening a subproof,
and introduced by shutting one,
has been adapted for lambda calculi.
Different modal logics may be encoded via different open and shut rules.
Prior work \cite{valliappan22} has given normalization proofs
for four Fitch-style formulations of lambda calculi with different modalities,
which required repeating the proofs for each individual calculus.
This prompts the need for a parametric Fitch-style modal lambda calculus
generalizing the variants,
in order to avoid repetition and ease further extensions.

\section{Background}

\textcite{fitch52} introduced a method for propositional deduction,
central to which is the idea of \emph{subordinate proof}.
For example, in order to prove $A \rightarrow B$ a subordinate proof may be opened
containing the new assumption $A$, where one sets out to prove $B$.
If successful, the subordinate proof can be shut
by introducing $A \rightarrow B$ in the original proof
thereby discharging the assumption $A$.

This has been adapted for modal lambda calculi \cite{borghuis94},
where modalities are type constructors that add some properties.
These may be understood using Kripke's possible worlds interpretation \cite{kripke63, huth04},
where opening a subproof means visiting a replica new world,
and shutting means returning.
To keep track of subordinate proofs in typing judgements
a new structural connective $\lock$ is added to the context when a $\Box$ is eliminated,
and popped when the subproof is closed.

\emph{Normalization by Evaluation} (NbE) is a technique for reducing terms to their normal forms,
which are not further reducible \cite{berger91}.
Instead of implementing the normalization procedure ``by hand'',
you instead proceed by evaluating,
before \emph{reifying} the resulting semantic value back into a term.
If at any point computation is blocked on a value known only at run time,
e.g. on an argument when descending into a lambda,
evaluation proceeds with a so called \emph{neutral value},
containing enough information about its origins to make reification possible.

Here we consider the family of Fitch-style modal lambda calculi
derived from intuitionistic propositional logic
extended with the unary connective $\Box$,
the inference rule \emph{necessitation},
if $\cdot \vdash A$ then $\cdot \vdash \Box A$,
and different axioms surrounding $\Box$ \cite{clouston18}.
The most basic modal calculi, $\lambda_\text{IK}$,
comes from the \emph{K axiom}
($\Box(A \rightarrow B) \rightarrow \Box A \rightarrow \Box B$).
K together with \emph{axiom T} ($\Box A \rightarrow A$) gives $\lambda_\text{IT}$;
K and \emph{axiom 4} ($\Box A \rightarrow \Box\Box A$) yield $\lambda_\text{IK4}$;
and K, T and 4 give $\lambda_\text{IS4}$.

\textcite{valliappan22} noted that for the four lambda calculi
$\lambda_\text{IK}$, $\lambda_\text{IT}$, $\lambda_\text{IK4}$, $\lambda_\text{IS4}$,
only the $\Box$-elimination rules differ,
see figure~\ref{fig:elim-rules},
and chose to instead define the $\Box$-elimination rules
in terms of different \emph{modal accessibility relations} $\Delta\lhd\Gamma$.
The relation $\Delta\lhd\Gamma$ should be thought as saying
``the contents of boxed values from context $\Delta$ may be accessed in the future context $\Gamma$,''
as will become apparent upon seeing the generalized $\Box$-elimination rule.
In this thesis, we use that concept to formulate
a single parametric calculus generalizing these four calculi.

\begin{figure}
  \begin{align*}
    &\inferrule[$\lambda_\text{IK}$/\Box-Elim]
    {\Gamma \vdash t : \Box A}
    {\Gamma, \Gamma' \vdash \unbox_{\lambda_\text{IK}} t : A}
    \lock \notin \Gamma' &
    &\inferrule[$\lambda_\text{IT}$/\Box-Elim]
          {\Gamma \vdash t : \Box A}
          {\Gamma, \Gamma' \vdash \unbox_{\lambda_\text{IT}} t : A}
          \#_{\lock} (\Gamma') \le 1 \\
          & \inferrule[$\lambda_\text{IK4}$/\Box-Elim]
            {\Gamma \vdash t : \Box A}
            {\Gamma, \lock, \Gamma' \vdash \unbox_{\lambda_\text{IK4}} t : A} &
            & \inferrule[$\lambda_\text{IS4}$/\Box-Elim]
            {\Gamma \vdash t : \Box A}
            {\Gamma, \Gamma' \vdash \unbox_{\lambda_\text{IS4}} t : A}
  \end{align*}
  \caption{$\Box$-elimination rules for the modal lambda calculi
    $\lambda_\text{IK}$, $\lambda_\text{IT}$, $\lambda_\text{IK4}$ and $\lambda_\text{IS4}$
    \cite{clouston18}.
    \label{fig:elim-rules}}
\end{figure}

The results have been formalized\footnote{Available online at
\url{https://github.com/axelf4/pfm-lambda}.}
in the proof assistant Agda \cite{norell07}.

\section{The calculus $\lambda_\text{PFM}$}

In this section we give the specification of the simply typed modal lambda calculus $\lambda_\text{PFM}$.
The calculus is parameterized by the binary relation $\Delta\lhd\Gamma$ on contexts,
subject to requirements that will be given below.

Types are constructed out of an uninterpreted base type $\iota$:
$$ \text{\emph{Type}} \quad A, B \Coloneqq \iota \mid A \to B \mid \Box A $$
Contexts are lists of types and locks:
$$ \text{\emph{Context}} \quad \Gamma \Coloneqq \cdot \mid \Gamma, A \mid \Gamma, \lock $$
The intrinsically typed syntax of the language is given in figure~\ref{fig:typing-rules},
where $\Gamma \vdash t : A$ is notation for $t$ being a well-typed term of type $T$ in context $\Gamma$.
De Bruijn indices are used to make $\alpha$-conversion implicit.

\begin{figure}
  \centering
  \begin{align*}
    &\inferrule[Var]{ }{\Gamma, x : A, \Gamma' \vdash x : A} \, \lock \notin \Gamma' &
    &\inferrule[\to-Intro]{\Gamma, A \vdash t : B}{\Gamma \vdash \lambda t : A \to B} &
    &\inferrule[\to-Elim]{\Gamma \vdash t : A \to B \\ \Gamma \vdash s : A}{\Gamma \vdash t \; s : B}
  \end{align*}
  \begin{align*}
    &\inferrule[\Box-Intro]{\Gamma, \lock \vdash t : A}{\Gamma \vdash \operatorname{box} t : \Box A} &
    &\inferrule[\Box-Elim]{\Delta \vdash t : A \\ \Delta \lhd \Gamma}{\Gamma \vdash \operatorname{unbox} t : A}
  \end{align*}
  \caption{The set of intrinsically typed terms of $\lambda_\text{PFM}$.
    The modal accessibility relation $\Delta\lhd\Gamma$ is a parameter of the calculus.
    \label{fig:typing-rules}}
\end{figure}

In order to present the equational theory
we define OPE:s and substitutions.
An \emph{order-preserving embedding} (OPE) is a binary relation on contexts $\Gamma \subseteq \Delta$
signifying $\Gamma$ can be weakened, i.e. add more assumptions,
to obtain $\Delta$.
It is defined inductively as
\begin{equation*}
  \inferrule{ }{\operatorname{base} : \cdot \subseteq \cdot} \quad
  \inferrule{\Gamma \subseteq \Delta}{\operatorname{weak} : \Gamma \subseteq \Delta, A} \quad
  \inferrule{\Gamma \subseteq \Delta}{\operatorname{lift} : \Gamma, A \subseteq \Delta, A} \quad
  \inferrule{\Gamma \subseteq \Delta}{\operatorname{lift_\lock} : \Gamma, \lock \subseteq \Delta, \lock}
\end{equation*}
We define a operation
$\operatorname{wk} : \Gamma\subseteq\Delta \to \Gamma \vdash t : A \to \Delta \vdash t : A$
that given an OPE weakens a term.
Only the case of weakening an $\operatorname{unbox}$ term is unlike the simply typed lambda calculus (STLC) counterpart:
$$ \textit{wk} \; w \; (\operatorname{unbox} \; t \; m) \coloneqq \operatorname{unbox} \; (\textit{wk} \; w' \; t) \; m' \quad \text{where } m' , w' = \textit{rewind}_\subseteq \; m \; w $$
where we require the calculus parameter
$$ \textit{rewind}_\subseteq : (m : \Gamma'\lhd\Gamma) \to (w : \Gamma\subseteq\Delta) \to \exists \Delta'. \, \Delta'\lhd\Delta \times \Gamma'\subseteq\Delta' $$
that given a modal accessibility relation $m$
truncates the contexts $\Gamma$ and $\Delta$ in $w$,
in order to remove as many locks from both as there are in $m$.
That is to say, it transports $w$ from the future world $\Gamma$ to instead act on the past world $\Gamma'$.

For substitutions - and later environments - we note that both
can be seen as replacement lists of items for each type in a context.
Thus we choose to define them as concrete instances of a type $\Rpl$,
parametric over some function
$F : \textit{Type} \to \textit{Context} \to \textbf{Set}$
and defined inductively as
\begin{equation*}
  \inferrule{ }{\cdot : \Rpl \cdot \; \Delta} \quad
  \inferrule{\sigma : \Rpl \Gamma \; \Delta \\ x : F \; A \; \Delta}
            {\sigma, x : \Rpl \; (\Gamma, A) \; \Delta} \quad
  \inferrule{\sigma : \Rpl \Gamma \; \Delta \\ m : \Delta\lhd\Delta'}
            {\operatorname{lock} \sigma \; m : \Rpl \; (\Gamma, \lock) \; \Delta'}
\end{equation*}
This helps unify some of the calculus parameters,
and avoids having some parameters depend on
e.g. the definition of terms which in turn depends on other parameters.
Substitutions may then be defined as $\operatorname{Sub} \coloneqq \Rpl$
with $F \; A \; \Gamma = \Gamma \vdash A$.

With the exception of the $\operatorname{lock}$ constructor
the definition of $\Rpl$ is as for substitutions in STLC.
Adding the alternate constructor
$\operatorname{lift}_\lock : \Rpl \Gamma \; \Delta \to \Rpl \; (\Gamma, \lock) \; (\Delta, \lock)$
it would be able to represent local substitutions in any of the ``worlds'' delimited by locks in the context.
Instead, $\operatorname{lock}$ with an argument $m : \Delta\lhd\Delta'$
(as used in \cite{valliappan22})
makes it possible to unify substitutions and the necessary \emph{modal transformations},
where locks are removed and added from contexts as permitted by $(\lhd)$.

With this choice of $\operatorname{lock}$ we make use of the parameter
$$ \lhd_\lock : \forall\Gamma. \, \Gamma \lhd \Gamma, \lock $$
in order to be able to define the identity substitution
$\textit{id}_s : \forall\Gamma. \, \operatorname{Sub} \Gamma \; \Gamma$.

As for OPE:s we define
$\textit{subst} : \operatorname{Rpl} \Gamma \; \Delta \to \Gamma \vdash t : A \to \Delta \vdash t : A$,
using the parameter
$$ \textit{rewind} : (m : \Gamma'\lhd\Gamma) \to (\sigma : \operatorname{Rpl} \Gamma \; \Delta) \to \exists \Delta'. \, \Delta'\lhd\Delta \times \operatorname{Rpl} \Gamma' \; \Delta' $$

The equational theory of $\lambda_\text{PFM}$ is given in figure~\ref{fig:eq-theory},
where reflexivity, symmetry, transitivity and congruence rules have been omitted.
The notation $\Gamma \vdash t \sim s$ says that the terms $t$ and $s$
are equal in the context $\Gamma$ up to the conversion relation $(\sim)$.

\begin{figure}
  \centering
  \begin{align*}
    \llap{$\beta$ equivalence:}& \quad
    \inferrule{\Gamma, A \vdash t : B \\ \Gamma \vdash s : A}{\Gamma \vdash (\lambda. \, t) \; s \sim t[s]} \quad
    \inferrule{\Delta, \lock \vdash t : A \\ m : \Delta \lhd \Gamma}{\Gamma \vdash \unbox \; (\operatorname{box} t) \; m \sim \textit{subst} \; (\operatorname{lock} \textit{id}_s \; m) \; t} \\
    \llap{$\eta$ equivalence:}& \quad
    \inferrule{\Gamma \vdash t : A \to B}
         {\Gamma \vdash t \sim \lambda. \, (\operatorname{wk} \; (\operatorname{weak} \textit{id}_\subseteq) \; t) \; (\operatorname{var} \operatorname{zero})} \quad
         \inferrule{\Gamma \vdash t : \Box A}{\Gamma \vdash t \sim \operatorname{box} \; (\operatorname{unbox} t \; \lhd_\lock)}
  \end{align*}
  \caption{Equational theory of $\lambda_\text{PFM}$.
    The rules for lambda abstraction are as for STLC.
    In the $\lambda$-$\beta$-conversion rule,
    $t[s]$ denotes applying a singleton substitution to $t$:
    Replacing the zeroth variable with $s$
    and decrementing all other de Bruijn indices.
    \label{fig:eq-theory}}
\end{figure}

\section{Normalization algorithm}

We provide a NbE algorithm based on a possible worlds model.
Normal and neutral forms are defined mutually as:
\begin{gather*}
  \inferrule{x : \text{Ne} \; \Gamma \; \iota}{\operatorname{ne} x : \text{Nf} \; \Gamma \; \iota} \quad
  \inferrule{x : \text{Nf} \; (\Gamma, A) \; B}{\operatorname{abs} x : \text{Nf} \; \Gamma \; (A \to B)} \quad
  \inferrule{x : \text{Nf} \; (\Gamma, \lock) \; A}{\operatorname{box} x : \text{Nf} \; \Gamma \; (\Box A)} \\
  \inferrule{x : A \in \Gamma}{\operatorname{var} x : \text{Ne} \; \Gamma \; A} \quad
  \inferrule{x : \text{Ne} \; \Gamma \; (A \to B) \\ y : \text{Nf} \; \Gamma \; A}
            {x \; y : \text{Ne} \; \Gamma \; B} \quad
  \inferrule{x : \text{Ne} \; \Gamma \; (\Box A) \\ m : \Gamma'\lhd\Gamma}{\operatorname{unbox} x \; m : \text{Ne} \; \Gamma \; A}
\end{gather*}
The normal forms are $\beta$-normal - no $\beta$-reductions are possible -
and $\eta$-long - all variables are maximally applied and unboxed,
as the $\operatorname{ne}$ constructor only permits neutral values of the base type.

As done in \cite{valliappan22}, we choose contexts for worlds,
weakenings for the intuitionistic accessibility relation between worlds, and
$(\lhd)$ for the modal accessibility relation.
(The intuitionistic accessibility relation should be thought of as
relating two worlds $w_1$ and $w_2$ if $w_2$ has as much or more knowledge than $w_1$;
for worlds as contexts this means all assumptions in $w_1$ should be present in $w_2$ too.)
Then we interpret types in the model as
\begin{equation}\label{eq:sem-values}
  \begin{split}
  \llbracket \iota \rrbracket_\Gamma &\coloneqq \text{Nf} \; \Gamma \; \iota \\
  \llbracket A \to B \rrbracket_\Gamma &\coloneqq \forall \Delta. \, \Gamma \subseteq \Delta \to \llbracket A \rrbracket_\Delta \to \llbracket B \rrbracket_\Delta \\
  \llbracket \Box A \rrbracket_\Gamma &\coloneqq \forall \Gamma', \Delta. \, \Gamma \subseteq \Gamma' \to \Gamma'\lhd\Delta \to \llbracket A \rrbracket_\Delta
  \end{split}
\end{equation}
and contexts as environments, i.e. lists of semantic values, using $\operatorname{Rpl}$,
$$ \llbracket \Gamma \rrbracket_\Delta \coloneqq \operatorname{Rpl}_{\llbracket\_\rrbracket} \Gamma \; \Delta $$
We have monotonicity for semantic values and environments,
i.e. we have
$wk_A : \Delta \subseteq \Delta' \to \llbracket A \rrbracket_\Delta \to \llbracket A \rrbracket_\Delta'$ and
$wk_\Gamma : \Delta \subseteq \Delta' \to \llbracket \Gamma \rrbracket_\Delta \to \llbracket \Gamma \rrbracket_\Delta'$.

The definition of evaluation, reification and reflection is given in figure~\ref{fig:nbe}.
The normalization function may then be given as
\begin{align*}
  &\textit{nf} : \Gamma \vdash t : A \to \operatorname{Nf} \; \Gamma \; A \\
  &\textit{nf} \; t \coloneqq \, \downarrow^A (\llbracket t \rrbracket \; \textit{id}_e)
\end{align*}
where $\textit{id}_e$ is the identity environment.
The algorithm can be summarized as follows:
Evaluation proceeds as for an interpreter,
except closures take an extra OPE argument -
this allows conjuring fresh variables under binders
when reifying functions.
Then the resulting semantic value is reified back to its normal form.
When reifying a box or function,
evaluation proceeds with the boxed term,
or the function applied to a neutral form corresponding to the argument type,
respectively.

\begin{figure}
  \centering
  \begin{align*}
    &\mathrlap{\text{Evaluation} \quad \llbracket\_\rrbracket : \Gamma \vdash t : A \to \forall\Delta. \, \llbracket\Gamma\rrbracket_\Delta \to \llbracket A \rrbracket_\Delta} &&\\
    &\llbracket x \rrbracket \; \hat\Gamma &&\coloneqq \text{lookup } x \text{ in } \hat\Gamma \\
    &\llbracket \lambda. \, t \rrbracket \; \hat\Gamma \; w \; \hat a &&\coloneqq \llbracket t \rrbracket \; (\operatorname{wk}_{\hat\Gamma} w \; \hat\Gamma, \hat a) \\
    &\llbracket t \; s \rrbracket \; \hat\Gamma &&\coloneqq \llbracket t \rrbracket \; \textit{id}_\subseteq \; (\llbracket s \rrbracket_{\hat\Gamma}) \\
    &\llbracket \operatorname{box} t \rrbracket \; \hat\Gamma \; w \; m &&\coloneqq \llbracket t \rrbracket \; (\operatorname{lock} \; (\operatorname{wk}_{\hat\Gamma} w \; \hat\Gamma) \; m) \\
    &\llbracket \operatorname{unbox} t \; m \rrbracket \; \hat\Gamma &&\coloneqq \llbracket t \rrbracket \; \hat\Delta \; \textit{id}_\subseteq m' \quad \text{where } m' , \hat\Delta = \textit{rewind} \; m \; \hat\Gamma \\
    &\mathrlap{\text{Reification} \quad \downarrow^A : \llbracket A \rrbracket_\Gamma \to \operatorname{Nf} \Gamma \; A} &&\\
    &\downarrow^\iota a &&\coloneqq a \\
    &\downarrow^{A \to B} a &&\coloneqq \operatorname{abs} \; (\downarrow^B \; (a \; (\operatorname{weak} \textit{id}_\subseteq) \; (\uparrow^A (\operatorname{var} \operatorname{zero})))) \\
    &\downarrow^{\Box A} a &&\coloneqq \operatorname{box} \; (\downarrow^A \; (a \; \textit{id}_\subseteq \lhd_\lock)) \\
    &\mathrlap{\text{Reflection} \quad \uparrow^A : \operatorname{Ne} \Gamma \; A \to \llbracket A \rrbracket_\Gamma} &&\\
    &\uparrow^\iota x &&\coloneqq \operatorname{nt} a \\
    &\uparrow^{A \to B} x \; w \; a &&\coloneqq \; \uparrow^B ((\operatorname{wk}_{\operatorname{Ne}} w \; x) \; \downarrow^A a) \\
    &\uparrow^{\Box A} x \; a &&\coloneqq \; \uparrow^B (\operatorname{unbox} \; (\operatorname{wk}_{\operatorname{Ne}} w \; x) \; m)
  \end{align*}
  \caption{Evaluation, reification and reflection definitions. \label{fig:nbe}}
\end{figure}

Here we have chosen the possible worlds inspired interpretation of $\Box A$,
instead of the syntax-directed approach of
$$ \llbracket \Box A \rrbracket_\Gamma = \llbracket A \rrbracket_{\Gamma, \lock}$$
one reason being that the $\unbox$ case of evaluation then would require
being able to apply the equivalent of a $\operatorname{lock} \textit{id} \; m$ substitution
on semantic values, or similar, in addition to weakening,
whereas currently no such thing is needed,
as the $m$ instead goes directly in $\unbox$ when reflecting.

\section{Correctness}

Soundness and completeness of the conversion relation
has been proved with respect to possible worlds,
with the addition of the following calculus parameters
\begin{itemize}
\item Rewinding $\operatorname{lock} \sigma \; m$
  with a modal accessibility relation $\Gamma \lhd \Gamma, \lock$
  should work as expected, i.e. give back $m$ and $\sigma$:
  $$ \textit{rewind-$\lhd_\lock$} : \forall m, \sigma. \, \textit{rewind} \lhd_\lock (\operatorname{lock} \sigma \; m) \equiv m , \sigma $$
  and the same for $\textit{rewind}_\subseteq$ on $\operatorname{lift_\lock}$.
\item The operation $\textit{rewind}$ should preserve composition:
  \begin{align*}
    \textit{rewindPres-$\circ$} &: (m : \Delta\lhd\Gamma) \to (\sigma_1 : \operatorname{Rpl} \Gamma \; \Gamma') \to (\sigma_2 : \operatorname{Rpl} \Gamma' \; \Gamma'') \\
    &\to
    \begin{array}[t]{@{}l@{}l@{}}
      \text{let } & m' , \sigma_1' = \textit{rewind} \; m \; \sigma_1 \\
      & m'' , \sigma_2' = \textit{rewind} \; m' \; \sigma_2 \\
      \text{in } & \textit{rewind} \; m \; (\sigma_1 \circ \sigma_2) \equiv m'' , \sigma_1' \circ \sigma_2'
    \end{array}
  \end{align*}
  and likewise for $\textit{rewind}_\subseteq$.
\item $\textit{rewind}$ should preserve identity:
  $$ \textit{rewindPresId} : (m : \Delta\lhd\Gamma) \to \textit{rewind} \; m \; \textit{id} \equiv m , \textit{id} $$
  and likewise for $\textit{rewind}_\subseteq$.
\item $\textit{rewind}$ should commute with weakening and substitution composition:
  \begin{align*}
    &\textit{rewindWk} &&: (m : \Delta\lhd\Gamma) \to (\sigma : \operatorname{Rpl} \Gamma \; \Gamma') \to (w : \Gamma' \subseteq \Gamma'') \\
    &&&\to
    \begin{array}[t]{@{}l@{}l@{}}
      \text{let } & m' , \sigma' = \textit{rewind} \; m \; \sigma \\
      & m'' , w' = \textit{rewind}_\subseteq \; m' \; w \\
      \text{in } & \textit{rewind} \; m \; (\textit{wk} \; w \; \sigma) \equiv m'' , \textit{wk} \; w' \; \sigma'
    \end{array} \\
    &\textit{rewindTrim} &&: (m : \Delta\lhd\Gamma) \to (w : \Gamma \subseteq \Gamma') \to (\sigma : \operatorname{Rpl} \Gamma' \; \Gamma'') \\
    &&&\to
    \begin{array}[t]{@{}l@{}l@{}}
      \text{let } & m' , w' = \textit{rewind}_\subseteq \; m \; w \\
      & m'' , \sigma' = \textit{rewind} \; m' \; \sigma \\
      \text{in } & \textit{rewind} \; m \; (\textit{trim} \; w \; \sigma) \equiv m'' , \textit{trim} \; w' \; \sigma'
    \end{array}
  \end{align*}
  where $\textit{wk} : \Delta\subseteq\Delta' \to \operatorname{Sub} \Gamma \; \Delta \to \operatorname{Sub} \Gamma \; \Delta'$ and
  $\textit{trim} : \Gamma\subseteq\Gamma' \to \operatorname{Sub} \Gamma \; \Delta \to \operatorname{Sub} \Gamma' \; \Delta$
  is substitution and weakening composition and vice versa, respectively.
\end{itemize}
These enable proving of the necessary weakening and substitution laws.
Indeed, most show up in the goals when proving the $\unbox$ cases of the laws.

\subsection{Completeness}

\begin{theorem}[Completeness]
  If $\Gamma \vdash t : A$, then $\Gamma \vdash t \sim \ulcorner \textit{nf} \; t \urcorner$.
\end{theorem}

The proof is an extension of the corresponding proof for STLC \cite{kovacs17},
and established by a \emph{Kripke logical relation} \cite{plotkin73}
between terms and semantic values:
\begin{align*}
  (\simeq^A) &\subseteq \Gamma \vdash A \times \llbracket A \rrbracket_\Gamma \\
  t \simeq^\iota \hat t &\coloneqq t \sim \ulcorner \hat t \urcorner \\
  t \simeq^{A \to B} \hat t &\coloneqq (w : \Gamma \subseteq \Delta) \to \forall a : \Delta \vdash A, \hat a : \llbracket A \rrbracket_\Delta.\,
  \operatorname{app} \; (\textit{wk} \; w \; t) \; a \sim \hat t \; w \; \hat a \\
  t \simeq^{\Box A} \hat t &\coloneqq (w : \Gamma \subseteq \Gamma') \to (m : \Gamma' \lhd \Delta)
  \to \operatorname{unbox} \; (\textit{wk} \; w \; t) \; m \sim \hat t \; w \; m
\end{align*}
It is \emph{logical} in the sense that terms and semantic values of box/function types
are related iff the results from unboxing/applying both to related terms and values, are related;
``Kripke'' means that we may first extend the context with a weakening.
Notice how the definition of the logical relation has the same shape
as that of semantic values, see equation \eqref{eq:sem-values}.
We will prove for each inductive step of normalization that $(\simeq)$ is maintained;
just as $\textit{nf}$ always returns a normal form through reification,
after reification we will get a $(\sim)$ out of $(\simeq)$.

We extend $(\simeq)$ to a relation $(\simeq_\text{ctx})$
between substitutions and environments elementwise related by $(\simeq)$,
and show that
the interpretations of a term in related substitutions and environments are related,
the so called \emph{fundamental theorem} of the logical relation:
\begin{lemma}[Fundamental theorem]
  If $t : \Gamma \vdash A$, $\sigma : \operatorname{Sub} \; \Gamma \; \Delta$,
  $\delta : \operatorname{Env} \; \Gamma \; \Delta$
  and $\sigma \simeq_\text{ctx} \delta$,
  then $\textit{subst} \; \sigma \; t \simeq \llbracket t \rrbracket \; \delta$.
\end{lemma}
\begin{proof}
  By induction on $t$.
  For the case of $t = \operatorname{box} \; s$, it needs to be shown that for all
  $w : \Delta \subseteq \Delta'$, $m : \Delta' \lhd \Xi$,
  \begin{equation*}
    \operatorname{unbox} \; (\operatorname{box} \; (\textit{wk} \; (\operatorname{lift}_\lock \; w) \; (\textit{subst} \; (\operatorname{lock} \; \sigma \; \lhd_\lock) \; s)))
    \simeq \llbracket s \rrbracket \; (\operatorname{lock} \; (\textit{wk} \; w \; \delta) \; m)
  \end{equation*}
  The induction hypothesis gives
  $$ \textit{subst} \; (\operatorname{lock} \; (\textit{wk} \; w \; \sigma) \; m) \simeq \llbracket s \rrbracket \; (\operatorname{lock} \; (\textit{wk} \; w \; \delta) \; m) $$
  which we compose with a conversion proof on the left
  \begin{equation*}
    \begin{split}
      &\Box\text{-}\beta \; (\textit{wk} \; (\operatorname{lift}_\lock \; w) \; (\textit{subst} \; (\operatorname{lock} \; \sigma \; \lhd_\lock) \; s)) \; m \\
      &\quad : \operatorname{unbox} \; (\operatorname{box} \; (\textit{wk} \; (\operatorname{lift}_\lock \; w) \; (\textit{subst} \; (\operatorname{lock} \; \sigma \; \lhd_\lock) \; s))) \\
      &\qquad \sim \textit{subst} \; (\operatorname{lock} \; \textit{id} \; m) (\textit{wk} \; (\operatorname{lift}_\lock \; w) \; (\textit{subst} \; (\operatorname{lock} \; \sigma \; \lhd_\lock) \; s))
    \end{split}
  \end{equation*}
  Here the term on the right side of $(\sim)$ does not immediately match up
  with the left side of $(\simeq)$ in the IH,
  however one can show that they are in fact equal
  using $\textit{rewind-/rewind$_\subseteq$-}\lhd_\lock$ and substitution laws.

  For the $t = \operatorname{unbox} \; s \; m$ case,
  it suffices to rewind the proof of $\sigma\simeq_\text{ctx}\delta$ by $m$ to get
  $$ \textit{rewind} \; m \; \sigma \simeq_\text{ctx} \textit{rewind} \; m \; \delta $$
  and apply the IH on it and $s$.
\end{proof}

Similarily, we show that reification and reflection also respect $(\simeq)$.
\begin{lemma}
  Reification and reflection respect $(\simeq)$:
  \begin{enumerate}
    \renewcommand{\theenumi}{\alph{enumi}}
  \item If $t : \Gamma \vdash A$, $\hat t : \llbracket A \rrbracket_\Gamma$
    and $t \simeq \hat t$ then $t \sim \ulcorner \downarrow\hat t \urcorner$.
  \item If $t : \operatorname{Ne} \Gamma \; A$ then $\ulcorner t \urcorner \simeq \, \uparrow t$.
  \end{enumerate}
\end{lemma}
\begin{proof}
  By induction on A.
  (We only show the $\Box A$ case of (a),
  with the rest left as an exercise to the reader.)
  The proof of $t \simeq \hat t$ says unboxing of $t$ and $\hat t$ respects $(\simeq)$.
  Choosing $w = \textit{id}_\subseteq$ and $m = \lhd_\lock$
  and applying the IH on the result yields
  $$ \operatorname{unbox} \; (\textit{wk} \; \textit{id}_\subseteq \; t) \; \lhd_\lock \simeq \ulcorner \downarrow (\hat t \; \textit{id}_\subseteq \; \lhd_\lock) \urcorner $$
  where $\textit{wk} \; \textit{id}_\subseteq \; t = t$.
  Combining this with the $\Box\text{-}\eta$ conversion rule for $t$, i.e.
  $$ t \sim \operatorname{box} \; (\operatorname{unbox} \; t \; \lhd_\lock) $$
  using transitivity and congruence under $\operatorname{box}$
  of the conversion relation gives the goal.
\end{proof}

The fundamental theorem,
using $\textit{id}_s$ and $\textit{id}_e$ for substitution and environment,
may then be combined with the reification lemma
to conclude the completeness proof.

One detail remains, however.
Depending on how $(\simeq_\text{ctx})$ is defined we may not be able to rewind it.
We know only how to rewind OPE:s and $\operatorname{Rpl}$:s,
not arbitrary data types, without taking more rewind functions as parameters.
Thus we let $(\simeq_\text{ctx}) \coloneqq \operatorname{Rpl}_{A\simeq\hat A}$ where
$$ A\!\simeq\!\hat A \; A \; \Gamma \coloneqq \exists t : \Gamma \vdash A, \hat t : \llbracket A \rrbracket_\Gamma.\, t \simeq \hat t $$
is collection of a term, a semantic value, and a proof that the two are related.
The substitution may be recouped by mapping the $\operatorname{Rpl}$
with a function that picks the $t$ out of each $A\!\simeq\!\hat A$,
likewise for the environment.
For the proof of the $\operatorname{unbox}$ case of the fundamental theorem,
we need two additional calculus parameters
\begin{itemize}
\item The transported modal accessibility relation
  should only depend on the contexts,
  with the actual contents of the $\operatorname{Rpl}$ being irrelevant:
  \begin{align*}
    \textit{rewindFree} &: (m : \Xi\lhd\Gamma) \to \forall \sigma : \operatorname{Rpl}_F \Gamma \; \Delta, \delta : \operatorname{Rpl}_G \Gamma \; \Delta.\, \\
    &\begin{array}[t]{@{}l@{}l@{}}
      \text{let } & m' , \_ = \textit{rewind} \; m \; \sigma \\
      & m'' , \_ = \textit{rewind} \; m \; \delta \\
      \text{in } & m' \equiv m''
    \end{array}
  \end{align*}
\item Rewind should commute with $\textit{map}_{\operatorname{Rpl}}$:
  \begin{multline*}
  \textit{rewindCommMap} : \forall (f : \forall A, \Gamma.\, F \; A \; \Gamma \to G \; A \; \Gamma), m, \sigma : \operatorname{Rpl}_F \; \Gamma \; \Delta. \\
  \textit{map} \; f \; (\pi_2 \; (\textit{rewind} \; m \; \sigma)) \equiv \pi_2 \; (\textit{rewind} \; m \; (\textit{map} \; f \; \sigma))
  \end{multline*}
  where $\pi_2$ is the second projection of the product type.
\end{itemize}
These enable us to get the rewound substitution/environment
from a rewound proof of $\sigma \simeq_\text{ctx} \delta$,
with $\textit{rewindFree}$ used to make the types agree.
It should be noted that we expect both equivalences to hold a priori,
as explained in \citetitle{wadler89} \cite{wadler89},
since a $\textit{rewind}$ instantiation cannot observe
the contents of the parametric $\operatorname{Rpl}$.
We still need to take them as parameters however,
in order to convince Agda.

\subsection{Soundness}

\begin{theorem}[Soundness]
  If $\Gamma : t \sim t'$, then $\textit{nf} \; t = \textit{nf} \; t'$.
\end{theorem}

The proof is accomplished by generalizing to a presheaf model
and proceeding along the same lines as in \cite{altenkirch95}.

\section{Remaining work}

Another point is to investigate whether to replace OPE:s with renamings,
i.e. substitutions where the replacement terms are variables.
Not only would this remove
the $\textit{rewind}_\subseteq$/$\textit{rewind}$ repetition,
but it would also be a step toward being able to add the \emph{R axiom}
($A \to \Box A$).
\textcite{valliappan-r} showed that the condition
$$ R_m \subseteq R_i $$
on $R_i$ and $R_m$,
the intuitionistic and modal accessibility relation, respectively,
is sufficient for ensuring axiom R is satisfied in the model
(with the IR \textsc{Var} rule and corresponding OPE definition).
Unlike with OPE:s as given here,
the condition holds when picking renamings for $R_i$,
since renamings encapsulate lock weakenings.

\printbibliography

\end{document}
